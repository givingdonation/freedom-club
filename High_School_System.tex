\documentclass{article}
\title{High School System: The state of clubs}
\author{Carlo Allietti}
\date{\today}
%\pagenumbering{gobble}
%\usepackage[bottom=0in]{geometry}
\begin{document}
\maketitle
\section{What's ``Freedom Club''?}
This is the question of everyone after I mention the club. Clearly it doesn't fit the high school club archetype. You know what speech and debate club does, they do speeches and debates. Science Olympiad does Science Olympiad. Archery shoots arrows. Democrats Club, well they do things relating to democrats. Freedom club does freedom? What is there to free at high school. There is no freedom, or at least no idea of what it is. Now let me be clear, I am not blaming people for not getting freedom club from the name alone, it's not clear on purpose, its very name was made to be unclear. To let me make the club malleable. But it does reflect something about what people expect of clubs. A goal or purpose is required to begin. Not just some group, even though that would allow for many degrees of freedom as compared to a purposeful club.

We know clubs are put into a rigid structure on purpose. You need a president, a vice president, a secretary, and a historian, because someone decided that sounds republican or ideal somehow. Then you need an advisor because brats can't be trusted, and you need to register it with the school, have its posters signed off, and use remind so the school can spy on us. You can't have a student run club, even though every club is student run, and you can't have an Instagram account with a non-CCSD teacher email because Instagram is incredibly dangerous. Same way every other little slice of freedom pie is taken from us. Perhaps I am just someone who likes to complain, but there is really no government in school, no advanced in A-Tech, not by AP Gov. standards, not by any. The hierarchies are reflected all the way down to the required structure of the clubs.
\section{Pasta Colander Insubordination}
Then when you do something to go against the school its pretty clear what the school does against you. How do we react? We oblige and oblige, no free thinking in sight. No politics, not big ones rallying politicians, not micropolitics and insurrection, the most has been is empty threats to contact the ACLU. Action begins when it first can, otherwise how do you prove you won't be just another brick in the wall. It doesn't need to be that much, clearly I won't expect you or myself to become Lenin, but maybe some group interested in doing something contrarian, or at least notably different might be good, or at least fun. There is no fun in the forms of fascism our society considers normal, which would be bureaucracy and adjacent politics. But I doubt many people will see the significance of a club without strict leaders/advisors/purposes, so maybe we need to start somehow. Maybe to fight fascism, we bring its different form into fruition, to see the monster firsthand, maybe that will make people care.
\section{Conclusion}
Perhaps we shouldn't lock the politics of the club just yet. Lets focus on its result for the school. Instead of a group of people arbitrarily related, like most CTSOs are, freedom club is more likely to foster real movement and real connections, to be a club that is actually student run. It will bring controversy to A-Tech, and not just the empty controversy of a club like European Student Union. If the club goes well, it will push people to fight for something, even if its a push against the club, at least its a movement of some kind, and it gets people thinking. For me the benefit is doing something I like the way I like it, a future for or strict benefit from a club like this is unlikely but still something I hope for. And if it doesn't work out then at least there was an attempt. And if it does work out, then my Fight Club fantasies sort of worked out, as delusional as they may seem. Either way, it will upset the state of clubs.

\end{document}
