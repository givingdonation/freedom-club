\documentclass{article}
\title{Primary Intentions}
\author{Carlo Allietti}
\date{August 24, 2024}
\pagenumbering{gobble}
\usepackage[bottom=0in]{geometry}
\begin{document}
\maketitle
\section{Preamble}
This document serves as a reflection of my attitudes at this current time. It is intended both for my own use, to reflect on how my ideas have evolved, and for others to understand the reasoning behind my actions. However, this document should not be seen as a definitive explanation of what my actions mean, as that is open to interpretation and may change. That being said, I have proceeded with creating this reflection.

\section{Situation}
It is required of me by society to go to school, and it is also required of me to follow its policies. What else can I do but complain and act if these policies do not align with what I believe is best for myself and others? The mere fact that something is required suggests there is a dogma that should be challenged if I hope to be an independent person. I am forced to tolerate a certain level of enforcement by the school of its goals, just as schools must tolerate a level of reaction to this enforcement. To sit back and let things happen is exactly what enables institutions to strip their subjects of control, whether you ultimately wish to preserve or destroy the institution.

Among the policies that I am opposing currently are those that constitute security theater. If a school justifies its policies by stating they are for security, then those policies should actually be effective. There are definitely deeper reasons behind policies such as dress code; the idea of dress code is likely and sometimes even stated to be primarily for restricting the culture and relationships of students, but those deeper reasons and their consequences or virtues are outside the scope of this document. For the purposes of security, what does a dress code accomplish? It does make it easier to identify students, but not any easier than tracking them and requiring id badges. Ultimately there is little that can be done for major threats to safety. It might be virtuous to try to do the little that might help in a serious situation, but you must remember that a balance between policy and freedom is what justifies the existence of all institutions.
\section{Conclusion}
The exact kind of practice I should undertake is not clear, nor does it need to be. Ideally, I should engage in something clever, which brings to mind a certain practice that has gained popularity: Pastafarianism. This movement is known for its members wearing pasta colanders on their heads as an inalienable form of religious expression. This is precisely the type of practice that could be viable for me.
\end{document}
